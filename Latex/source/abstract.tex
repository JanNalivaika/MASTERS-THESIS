% !TeX spellcheck = en_US
% In total max. 1 Page!
\IWBstudentthesisAbstract{%
%
% Abstract English:
This thesis discusses a methodology for analyzing and optimizing process variables in manufacturing systems with redundant \acrshort{DoF}s. The main focus lies on a 6-\acrshort{DoF} industrial robot that traverses a toolpath defined in 5-\acrshort{DoF}s (X,Y,Z,A,B). This means that the rotation C, which represents the rotation around the tool symmetry axis, is a redundant \acrshort{DoF} and can be chosen freely. The first part of the methodology covers the analysis of process variables, such as direction changes in the joints or the total travel of the joints. The goal is to understand how the redundant \acrshort{DoF}s affect the robot's behavior and how they can be used to optimize user-selected process variables. The second part of the methodology proposes a procedure to optimize the user-selected process variables by properly constraining the redundant \acrshort{DoF}. After modeling the robot in Python, validation is performed on multiple toolpaths. The results show that the redundant \acrshort{DoF}s can significantly affect the process variables. By implementing a \acrshort{PSO} algorithm, it is possible to find the ideal setting for the redundant \acrshort{DoF}s, thus maximizing the process quality with respect to the process variables. 
}
{%
	%
	% Zusammenfassung Deutsch:
	In dieser Arbeit wird eine Methodik zur Analyse und Optimierung von Prozessvariablen in Fertigungssystemen mit redundanten Freiheitsgraden diskutiert. Der Schwerpunkt liegt auf einem Industrieroboter mit sechs Freiheitsgraden, der eine in Werkzeugpfad abfährt, der nur fünf Freiheitsgrade (X, Y, Z, A, B) definiert. Dies bedeutet, dass die Rotation um Achse C, die die Werkzeugsymmetrieachse darstellt, ein redundanter Freiheitsgrad ist und frei gewählt werden kann. Der erste Teil der Methodik umfasst die Analyse von Prozessvariablen wie Richtungswechsel in den Gelenken oder der Gesamtverfahrwege der Gelenke. Das Ziel ist es, zu verstehen, wie sich die redundanten Freiheitsgrade auf das Verhalten des Roboters auswirken und wie sie zur Optimierung der vom Benutzer ausgewählten Prozessvariablen verwendet werden können. Der zweite Teil der Methodik schlägt ein Verfahren zur Optimierung der ausgewählten Prozessvariablen vor. Nach der Modellierung des Roboters in Python wird die Validierung anhand mehrerer Werkzeugbahnen durchgeführt. Die Ergebnisse zeigen, dass die redundanten Freiheitsgrade die Prozessvariablen erheblich beeinflussen können. Durch die Implementierung eines \acrshort{PSO}-Algorithmus ist es möglich, die ideale Einstellung für die redundanten Freiheitsgrade zu finden und somit die Prozessqualität in Bezug auf die Prozessvariablen zu maximieren.
	
	
}%
%
%
