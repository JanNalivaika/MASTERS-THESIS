\chapter{Appendix}%

\begin{comment}
	This thesis discusses a method for analyzing process variables and optimizing boundary conditions in manufacturing systems with redundant degrees of freedom (DoF). Strong focus is placed on industrial robots that traverse a toolpath in 5-DoF. This means that while the toolpath defines the X, Y, and Z coordinates, as well as rotations A and B, rotation C is not defined and can be chosen freely. The first part of the methodology covers the analysis of process variables such as direction changes in the joints and total travel of the joints, as well as the influence of the defined boundary condition that constrains the redundant DoF to a specific value.
	The second part of the method proposes a procedure to optimize the setting of the redundant DoF in order to optimize the user-selected process variables. The user also has the option to specify an importance rating for each process variable, thus describing the process quality numerically. This problem is analogous to an optimization problem where the process quality is to be maximized. It is shown that the definition of rotation C can have a significant impact on the process variables.
	Furthermore, in order to analyze the different process variables when 2 redundant DoF are present, a rotary-tilt table is added to the analysis. Only the tilting aspect is analyzed. It is shown how the process quality changes significantly depending on the combination of these 2 redundant DoF.
	The last part of the validation involves implementing a PSO algorithm to find the best possible setting for both redundant DoF, thus maximizing the process quality. This procedure shows that it is not necessary to analyze each possible combination of the redundant DoF in a brute-force manner. The optimization algorithm converged to the global optimum in 5 iterations with a population size of 20.
\end{comment}