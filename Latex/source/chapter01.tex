% !TeX spellcheck = en_US
\chapter{Introduction}%
\section{Motivation}%

In the age of "Industrie 4.0", advanced technologies like digital twins, have greatly transformed industrial manufacturing~\cite{Singh.2021}. A considerable amount of data can be gathered from various processes, like milling or 3D printing. By analyzing this data, it is possible to find new and optimized methods for efficient manufacturing~\cite{Ghobakhloo.2020}. By doing so, a significant amount of resources, like time and money, can be saved while at the same time increasing the quality of the produced product~\cite{Bibby.2018,Simonis.2016}.\newline
Computer-Aided Manufacturing (CAM) has been introduced as a crucial tool to improve productivity and accuracy in creating customized products~\cite{Feldhausen.2022}. CAM systems automate and optimize tasks such as machining, welding, and assembly~\cite{LalitNarayan.2013b}. One of the key strengths of CAM lies in its precision and consistency, ensuring that intricate components are produced with minimal error. Furthermore, CAM systems contribute to increased efficiency by minimizing material waste and reducing production time~\cite{Dubovska.2014}. These capabilities play a significant role in achieving a carbon-neutral production process~\cite{Saxena.2020}. One of the most important areas of CAM is the calculation of the tool path for computer numerical control (CNC) machines as well as the movement and behavior of multi-axis industrial robots~\cite{Pan}. \newline


Manufacturing machines are the backbone of modern industrial processes~\cite{Bi.2020}. These machines encompass a wide range of equipment, from CNC machining centers to 3D printers and automated assembly lines. Their primary ability lies in precision and efficiency. CNC machines, for instance, can repeatedly produce intricate parts with high accuracy, reducing human error and ensuring consistency~\cite{Jia.2018}. \newline
Industrial robots are a dominant part in the area of manufacturing as they can perform multi-axis movements that are needed to fulfill the customers wishes for individualized products~\cite{Sherwani.2020}. They are cheaper to acquire and more flexible compared to CNC milling machines, but have their own set of disadvantages~\cite{Iglesias.2015}. One of the most important advantages is its wide adaptability. They allow for quick reconfiguration to produce different components or products, promoting flexibility in manufacturing~\cite{Billard.2019}. Further, advancements in robotics and artificial intelligence (AI) have broadened their capabilities, enabling tasks that were once deemed too complex or hazardous for humans~\cite{Goel.2020}. \newline

Achieving efficiency and sustainability in the current fast-changing environment requires a thorough analysis of the interdependent relationships between the manufactured part, process parameters, and boundary conditions that govern multi-axis robot programs~\cite{Pan, Gadaleta.2019}. As the companies that work with industrial robots can place a strong emphasis on energy reduction, cycle-time minimization, or precision, optimizing these parameters is essential. CAM enables the simulation of the planned process, thus adapting any boundary conditions to fit the selected goals~\cite{Kyratsis.2020,Maiti.2017,Pan,Uhlmann.2016}.
This thesis is focused on a methodical approach for analyzing process parameters and optimizing boundary conditions in multi-axis robot programs. 

\section{Problem Formulation}\label{Problem Formulation}
Manufacturing systems that incorporate redundant degrees of freedom (DoF) offer significant advantages in terms of flexibility and adaptability~\cite{Anjum.2022}. One example of a system with redundancy is a 6-DoF industrial robot with a rotary tilt table, which brings the system to eight DoF. However, these systems also present various conflict points that need to be carefully managed to ensure optimal performance~\cite{Boscariol.2020, Liu.2022}.


One of the critical challenges in manufacturing systems with redundant DoF is singularity avoidance~\cite{Liu.2022}. Singularities occur when the robot manipulator loses control or achieves limited mobility due to certain configurations~\cite{Malyshev.2022}. These configurations result in the loss of a DoF or make the system highly sensitive to small changes, leading to unstable or unpredictable behavior~\cite{Zhao.2021, Milenkovic.2021}. Limiting the possible positions by adding artificial constraints can help to avoid this problem~\cite{Faria.2018}. %Advanced control algorithms and motion planning techniques are necessary to detect and avoid singularities, ensuring the system operates within safe and stable regions.

%Redundant degrees of freedom introduce complexities in controlling joint accelerations and jerks. Rapid changes in acceleration and jerk can cause mechanical stress, decrease system lifespan, and compromise precision. Proper trajectory planning and control schemes that consider joint acceleration and jerk limits are crucial to maintain smooth and controlled motion, reducing the risk of mechanical failures and ensuring accurate positioning.

%Extension:
%Redundant degrees of freedom can provide additional extension capabilities to industrial robots, allowing them to reach difficult-to-access areas. However, managing and controlling the extension can be challenging, particularly when precise positioning or maintaining stability is required. Accurate modeling, control algorithms, and calibration techniques are necessary to ensure proper extension control and to avoid issues related to overextension or underextension.

%Energy Use:
%Redundant degrees of freedom can increase energy consumption in manufacturing systems. Additional joints and actuators require more power, leading to higher energy costs and environmental impact. Efficient energy management strategies, such as optimizing control algorithms, utilizing regenerative braking, and incorporating energy-efficient components, are crucial to minimize energy use while maintaining system performance.

%Direction Changes:
%Redundant degrees of freedom offer increased flexibility in changing the direction of motion. However, abrupt and frequent direction changes can lead to mechanical stress, decreased precision, and increased energy consumption. Proper trajectory planning, smooth path generation, and optimized control algorithms are essential to manage direction changes effectively, ensuring smooth and controlled motion without compromising system performance.

%Transfer Time:
%Transfer time is another critical aspect affected by redundant degrees of freedom. Complex motion patterns required for utilizing redundant degrees of freedom can increase the time required to complete a task. Minimizing transfer time is crucial for improving production efficiency and throughput. Optimal path planning, motion optimization, and parallel processing techniques can be employed to reduce transfer time while leveraging redundant degrees of freedom effectively.

%Precision:
%Redundant degrees of freedom can enhance precision and accuracy in manufacturing systems. However, achieving and maintaining high precision can be challenging due to increased complexity and sensitivity to various factors. Calibration techniques, accurate modeling, feedback control, and error compensation algorithms are vital to ensure precise positioning and improve overall system accuracy.

%Maximum Load Capacity:
%Redundant degrees of freedom can influence the maximum load capacity of industrial robots. Increased complexity and additional joints can affect the robot's ability to handle heavy loads or maintain stability. Proper design considerations, such as structural reinforcement, joint torque and force limitations, and payload distribution, are necessary to ensure safe and reliable operation under maximum load conditions.

%Stiffness:
%The presence of redundant degrees of freedom can affect the stiffness of the manufacturing system. Increased flexibility due to redundant degrees of freedom may introduce compliance and reduce overall system stiffness. This can impact precision, accuracy, and stability. Careful design, utilization of appropriate materials, and control strategies that compensate for compliance can be employed to ensure the desired level of stiffness and system rigidity.


%---------------





One significant aspect of manufacturing systems with redundant DoF is joint acceleration and jerk, which is the rate of change of acceleration. The robot must allocate accelerations effectively among its joints to achieve smooth and coordinated motion. Failure to do so can result in jerky or erratic movements, which not only compromise precision but also impact the efficiency of the manufacturing process~\cite{Duong.2021}. Rapid changes in acceleration and jerk can cause mechanical stress, decrease system lifespan, and compromise precision. Additionally, the joints can be limited in their ability to keep up with the required speed due to limitations in power~\cite{Staff.1988}. Therefore, advanced control algorithms and motion planning techniques are necessary to optimize joint motion and minimize conflicts in joint acceleration and jerk~\cite{Duong.2021, Valente.2017}.

Extension control is another critical aspect that needs to be addressed in systems with redundant DoF. Redundant DoF can provide additional extension capabilities to industrial robots, allowing them to reach difficult-to-access areas~\cite{Duong.2021}. However, managing and controlling the extension can be challenging, particularly when precise positioning or maintaining stability is required~\cite{Lin.2022}.
The robot must accurately determine the appropriate position for each joint to avoid unnecessary overextension and collisions with the surrounding environment. The robot pose, which is the combination of position and orientation in three-dimensional space, also has a significant effect on robot stiffness~\cite{Xiong.2019}. An increased number of joints can introduce more play and reduce overall system stiffness. This can affect precision, accuracy and stability. Robot pose and its DoF must be carefully considered to ensure the desired level of system rigidity~\cite{Liu.2022, Shi.2021}.


Precision is a crucial element in manufacturing systems, closely tied to its stiffness. The robot needs to have precise control over the movement of each joint to achieve the desired accuracy of position in the manufacturing process. Nevertheless, achieving and maintaining high accuracy and repeatability can be difficult due to the increased complexity and sensitivity to various factors~\cite{Duong.2021}. %Any deviations or errors in joint motion can lead to compromised precision and reduced product quality. %To address this, advanced sensing and feedback control mechanisms, such as force/torque sensors and visual feedback systems, can be employed to enhance precision and ensure the accurate execution of manufacturing tasks.
Frequent changes in direction in the joints are another factor that affects precision. 
Due to the serial kinematics of industrial robots, the present play in the motor joints can add up the inaccuracies and impede the manufacturing process~\cite{Huynh.2020, ChenGang.2014}. Mechanical stress, decreased precision, and increased energy consumption can result from abrupt and frequent direction changes~\cite{Gasparetto.2010}.
 
Furthermore, effectively coordinating the movement of multiple joints to execute rapid direction changes can prove to be a difficult and computationally intensive task~\cite{VandeWeghe.2007}. Poor direction changes can result in prolonged and unnecessary movement times, ultimately hampering the overall productivity of the manufacturing process~\cite{Reiter.2016}. %Therefore, advanced control algorithms and motion planning techniques are required to optimize direction changes and minimize transfer time, ensuring efficient and timely execution of manufacturing tasks.
Minimizing production time is crucial for improving efficiency and throughput. Optimal path planning, motion optimization, and parallel processing techniques can be employed to reduce non-value adding movements while leveraging redundant DoF effectively~\cite{Boscariol.2020}.




Energy use is also a significant concern in manufacturing systems employing redundant DoF~\cite{Doan.2016}. The presence of additional joints and their non-optimal usage can require more power to operate, potentially leading to increased energy consumption. As energy efficiency becomes a priority in modern manufacturing, efficient energy management strategies are necessary to mitigate the increased power demand~\cite{Boscariol.2020, Boscariol.2019}. 


%Maximum load capacity is another area influenced by redundant degrees of freedom. The presence of additional joints can introduce structural limitations, affecting the robot's ability to handle heavy loads. Structural analysis and design considerations are crucial to ensure that the robot's mechanical structure can withstand the required load capacity without compromising its integrity or performance. This may involve using high-strength materials, optimizing structural configurations, and employing robust safety mechanisms to prevent overloading.

While redundant DoF may introduce potential conflicts and require special attention, they can also significantly enhance performance in manufacturing systems~\cite{Ayten.2016}. The added DoFs increase flexibility and adaptability, enabling the robot to carry out complex tasks more efficiently. Redundancy enables multiple approaches to achieve a desired end-effector position or orientation. By effectively utilizing the surplus of DoF, manufacturing systems can enhance their performance, increase efficiency, and exhibit greater flexibility in handling diverse tasks~\cite{Boscariol.2020}. 

Currently, there is no integrated system that can evaluate a computed tool path based on the chosen objective, such as minimizing movement or maximizing stiffness, in conjunction with available CAM systems. Additionally, there is no option to provide an optimal or near-optimal solution for defining the necessary constraints for a specific goal like for example, minimizing energy usage.


\section{Objective}%
The definition of the redundant constraints, mentioned in chapter \ref{Problem Formulation}, does not affect the relative tool path as generated by the CAM software. As such, a methodical approach to optimize these constraints without altering the toolpath in terms of efficiency, speed, and energy demand of the machine is required. Currently, no literature provides a comprehensive analysis or methodology regarding this global optimization problem.
This work aims to attain a methodical approach that analyzes a set of constraints and evaluates the influence of those constraints on a set of defined process variables. This work is focused on a 6-axis robot with a rotary-tilt table, whereby the results should also be transferable to other machines. Furthermore, the experiments and validations will be limited to the manufacturing processes of WAAM and milling. 


First, the influence of the constraints on relevant process variables (energy demand, joint turnover, speed and acceleration peaks, total joint movements) in a manufacturing process such as WAAM will be assessed. Subsequently, a process evaluation will be elaborated in the CAM software, by means of which the process quality can be determined. Depending on the respective process variables, approximation methods or machine learning methods will be investigated for the process evaluation. The process quality as a one-dimensional variable will be determined by weighting the process variables. Subsequently, a method for the optimization of the constraints will be elaborated. This task corresponds to an optimization problem in which the process quality will be maximized by selecting suitable constraints. 
