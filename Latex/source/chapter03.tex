% !TeX spellcheck = en_US
\chapter{Methodology}%

\section{Introduction}%

The proposed method aims to provide a framework for optimizing various parameters of an industrial robot. By effectively utilizing redundant degrees of freedoms mentioned in in Chapter \ref{OBJECTIVE}, this method is applicable to robotic milling operations and WAAM processes. The successful implementation of this methodology improves the robot's overall performance and efficiency, leading to increased productivity in industrial operations.

The first step involves constructing a basic model that captures the kinematics and dynamics a the industrial robot. To test the method in a simple case, the first tests are performed on a non-redundant 6-DoF model. After validation on this simple model, redundant degrees of freedom are introduced. After that the method is connected with CAM software to be tested in more complex scenarios.

Once the mathematical model is established, various optimization algorithms are implemented to determine the optimal values for each parameter associated with the redundant degrees of freedom. These method and optimization algorithms will consider the industrial robot's specific objectives and constraints, like energy consumption, feed rates and accelerations.
The validation process entails conducting real-world tests and simulations to evaluate optimized parameter performance and verify the effectiveness of our proposed methodology.


\section{General Methodology}
