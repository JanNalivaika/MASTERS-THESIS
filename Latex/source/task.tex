% Deckblatt
\chapter*{Scope of Work}

%\markright{Aufgabenstellung} 	% Kolumnentitel manuell auf "Aufgabenstellung"

\textbf{Title of the Master's Thesis:}\\
\Large{Methodical Approach for Analyzing Process Variables and Optimizing Boundary Conditions in Multi-Axis Robot Programs}\\
\newline
\normalsize{\textbf{Titel der Master's Thesis:}}\\
\Large{Methodischer Ansatz zur Analyse von Prozessvariablen und Optimierung von Randbedingungen in Multi-Achs-Roboterprogrammen}
\normalsize

\begin{tabbing}
	\hspace{7em} 		\= \hspace{13em}			\= \hspace{7em} 		\= \kill
				\> 						\> \\
	\textbf{Author:} \>  Jan Nalivaika	\> \textbf{Supervisor:} 	\>Ludwig Siebert  \\
	\textbf{Issue:} 	\> 02.10.2023	\> \textbf{Submission:} 	\> 01.03.2024
\end{tabbing}

\vspace{5mm}
\textbf{Motivation:}\\
Computer-aided manufacturing (\acrshort{CAM}) is used to automatically generate tool paths for 
computer numerically controlled machines. The \acrshort{CAM} software considers the models of the 
raw and finished parts, the constraints of the machine, the tools, and the manufacturing 
technology. Together with user-configurable parameters, tool paths for 3-axis, 5-axis, or 
robot-based machine tools are generated. The growing demand for flexibility in machine tools, 
such as the use of multiple manufacturing technologies in one machine or automated loading 
and unloading, has led to many machine tools being equipped with additional mechanical 
axes. Examples include robots mounted on linear axes and rotary-tilt tables.
The tool paths created in \acrshort{CAM} programs are usually defined by five degrees of freedom. The 
first three are the translational axes X, Y, and Z. The tilting and inclining of the tool are defined 
by the A- and B-axes. Occasionally, an additional rotation of the tool (C-axis) around the Z-axis (e.g., for dragging a swivel knife) is defined. Machines with more degrees of freedom than 
those limited by the toolpath often need user-defined constraints. These constraints are 
necessary to fully specify the movements of the machine axes. An example is the alignment 
of a part using the rotary-tilt table so that the Z-axis of the tool always points in the direction of 
gravity. This is helpful in processes like fused deposition modeling (\acrshort{FDM}) and wire arc additive 
manufacturing (\acrshort{WAAM}).
It is common practice to set the user-defined constraints based on experience. The definition 
of these constraints does not affect the relative tool path generated by the \acrshort{CAM} software. A 
preliminary literature review indicates that the configuration of these degrees of freedom has 
an impact on the energy demand and stability of the process. As such, a methodical approach to optimize these constraints in terms of 
efficiency, speed, and energy demand of the machine is required. Currently, no literature
provides a comprehensive analysis or methodology regarding this global optimization 
problem.

\vspace{5mm}
\textbf{Objective:}\\
This work aims to attain a methodical approach that analyzes a set of constraints and 
evaluates the influence of those constraints on a set of defined process variables. It will focus
on a 6-axis robot with a rotary-tilt table, whereby the results should also be transferable to other machines. Furthermore, the experiments and validations will be limited to the 
manufacturing processes of \acrshort{WAAM} and milling.
First, the influence of the constraints on relevant process variables (energy demand, joint 
turnover, speed and acceleration peaks, and total joint movements) in a manufacturing 
process such as \acrshort{WAAM} will be assessed. Subsequently, a process evaluation will be
elaborated in the \acrshort{CAM} software, by means of which the process quality can be determined.
Depending on the respective process variables, approximation or machine learning methods 
will be investigated for the process evaluation. The process quality as a one-dimensional 
variable will be determined by weighting the process variables. Subsequently, a method for 
the optimization of the constraints will be elaborated. This task corresponds to an optimization 
problem in which the process quality will be maximized by selecting suitable constraints.\newline


\vspace{5mm}
\textbf{Procedure and working method:}\newline

The following work packages are conducted within this thesis:

\begin{itemize}
\item Literature research
\item Familiarization with \acrshort{WAAM}, milling machines, and \acrshort{CAM} software
\item Selection of suitable process variables
\item Elaboration of the proposed method in a suitable programming language
\item Verification and validation of the elaborated method
\item Documentation of the work

\newpage	 
\end{itemize}
\vspace{1.0cm}
\textbf{Agreement:}\\

Through the supervision of B.Sc. Jan Nalivaika intellectual property of the \textit{iwb} is incorporated in this work. Publication of the work or transfer to third parties requires the permission of the chair holder. I agree to the archiving of the work in the library of the \textit{iwb}, which is only accessible to \textit{iwb} staff, as inventory and in the digital student research project database of the \textit{iwb} as a PDF document.
\vfill

\IWBaddressCityChair, 02.10.2023
\vspace{2.5cm}\\
\begin{tabular}{p{0.5\linewidth}p{0.5\linewidth} }
	Prof. Dr.-Ing.		& B.Sc.\\
Michael F. Zäh  	& Jan Nalivaika
\end{tabular}
