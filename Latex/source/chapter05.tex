% !TeX spellcheck = en_US
\chapter{Conclusion}%

\section{Summary}%


\newpage

\section{Outlook}%

This work focuses on a select number of process parameters, but this selection can be expanded to further optimize the real-world manufacturing process.

One potential additional parameter that can be optimized, is the stiffness value of the robotic system based on its current pose. Maintaining high stiffness in the orthogonal cutting direction is crucial for minimizing deviations, while low stiffness combined with high contact forces can result in significant dimensional errors. Finite-element analysis or multi-body simulation can be used to determine the stiffness value and optimize it for better performance by specifically defining the redundant DoF.

Another important parameter for future research is the collision index. The collision index is used to identify potential collisions between any part of the robot or the end-effector and the workpieces or other objects in the environment. This parameter is particularly significant in scenarios involving WAAM systems, where loose wires can change their position depending on the robots pose. Preventing collisions is essential to avoid damage to the workpiece, the robot, or other equipment. As of now the optimization algorithm does not consider this option.

One area where the PSO optimization process can lead to significant advantages is section-wise optimization. The toolpath can be divided into sections, and for each section, the optimal settings of the redundant DoF can be determined. To make this approach work effectively, it is important to consider the boundary conditions at the transition points between the individual sections.

Furthermore, implementing the proposed method in a CAM software can provide faster computation and enable the optimization of more complex toolpaths. Another option to speed up the computation is to program the algorithm in such a way that it is designed for multi-threading. Such implementation can result in significantly faster optimization.  

Moving forward, it is necessary to conduct further validation processes, including real-world tests and simulations, to evaluate the performance of the optimized parameters and verify the effectiveness of the proposed methodology. Additionally, exploring the implementation of machine learning techniques for best-case calculations could be a valuable avenue for future research.

In conclusion, conducting additional tests and thorough validation can greatly enhance the efficiency, productivity, and safety of the manufacturing process.