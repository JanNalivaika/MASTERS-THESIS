% !TeX spellcheck = en_US
\chapter{Conclusion}%

\section{Summary}%

\section{Outlook}%
		Cycle time 
		%Another important parameter is the cycle time, which represents the time required to execute the G-code. Various factors can influence the cycle time, such as feed rates, engagement depth of a cutting tool, or wire feed rates in the case of additive manufacturing processes. By adjusting these parameters, manufacturers can significantly reduce or increase the cycle time, optimizing the efficiency and productivity of the manufacturing process.
		
		%Analyzing and optimizing the cycle time helps to streamline operations, reduce production time, and enhance overall productivity. It allows manufacturers to identify opportunities for improvement and make informed decisions regarding process parameters, leading to increased throughput and cost savings.
		
				Stiffness value
				
				%Another crucial factor in industrial robot tool path execution is the stiffness value. This value specifically refers to the stiffness in the direction of the highest contact forces, or the most critical direction. In certain machining scenarios, such as when cutting a slot with full engagement, the stiffness in the cutting direction may be of lesser importance compared to the stiffness in the perpendicular direction of the cut.
				
				%Maintaining high stiffness in the orthogonal cutting direction is vital, as it helps minimize deviations from the desired mid-axis of the slot. Conversely, when high contact forces are combined with low stiffness, it can result in significant deviations in the final dimensions of the machined part.
				%To determine the stiffness value, a finite-element analysis, multi-body simulation or CAM simulation can be employed. These simulations provide the necessary data, which is also stored in the form of an array, to extract the stiffness values for analysis and optimization purposes.
				
				
						Collision index

%The collision index is used to determine if any potential collisions could occur between any part of the robot or the end-effector and the workpieces or other objects in the environment. This parameter is particularly important in scenarios involving WAAM systems, where loose wires can change their position depending on the robot's pose. It helps to prevent potential collisions that could lead to damage to the workpiece, the robot, or other equipment.

Man kann im Ausblick zeigen, dass ein Bauteil-Verschieben auch eine Verbesserung bringen kann, aber das ist erstmal nicht der Fokus