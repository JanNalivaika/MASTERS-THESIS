% !TeX spellcheck = en_US
\chapter{Conclusion}%

\section{Summary}%
This thesis proposes a method for optimizing the execution of a toolpath on robotic systems with redundant \acrshort{DoF}.

Firstly, the problem formulation highlights the flexibility and various issues that arise with redundant \acrshort{DoF}. It discusses how singularity avoidance can help industrial robots avoid sub-optimal poses that may lead to unexpected behavior. Redundant \acrshort{DoF} can also affect joint accelerations and jerk, potentially causing excessive wear on parts and resulting in more downtime and sub-optimal part quality. Additionally, other factors such as extension control, precision, and energy use are mentioned, and their impact on the manufacturing process is discussed.

The aim of this thesis is to propose and validate a systematic approach to leveraging these redundant \acrshort{DoF} in order to optimize towards a user-defined goal. Currently, there is no publication available that presents a general solution to this problem.


Chapter \ref{SST} provides a comprehensive discussion on the state of science and technology, aiming to provide a clear understanding of the individual components of manufacturing systems and optimization algorithms. Both subtractive and additive manufacturing are examined in detail, including a review of their respective strengths and weaknesses. An important focus is placed on one of the common processes in additive manufacturing, namely \acrshort{WAAM}, which has a significant relationship with industrial robots. In Chapter \ref{IR}, the functionality of these robots is described in detail, with special attention given to the issue of redundancy in such robotic systems.

Chapter \ref{comp} focuses on a comparative analysis of published research papers related to singularity avoidance, optimization of joint acceleration and jerk, optimization of energy use, and optimization of stiffness. Each section highlights the available options and approaches for leveraging redundant \acrshort{DoF} to achieve improved performance in these specific areas. The examination of multiple methods serves as the foundation for understanding the current state of cutting-edge research and identifying any existing research gaps.

Thus far, no global optimization approach has been proposed that can consider a user-defined input with specified goals, based on multiple process parameters, and providing the optimal settings for the redundant \acrshort{DoF}. This represents an important research opportunity in the field.

Chapter \ref{METmain} presents a solution for the identified research gap. In order to address this problem, a carefully selected list of process parameters is introduced and thoroughly discussed. These parameters are derived from the movement of the robot arm and include variables such as the rotational position of joints over time, as well as their subsequent derivatives and direction changes.

After summarizing which process parameters can be extracted form a toolpath traversed by a industrial robot, the first main step of the methodology is presented. To calculate the global score it is necessary to calculate the individual local score 

\section{Outlook}\label{Outlook}

This work focuses on a select number of process parameters, but this selection can be expanded to further optimize the real-world manufacturing process.

One potential additional parameter that can be optimized, is the stiffness value of the robotic system based on its current pose. Maintaining high stiffness in the orthogonal cutting direction is crucial for minimizing deviations, while low stiffness combined with high contact forces can result in significant dimensional errors. Finite-element analysis or multi-body simulation can be used to determine the stiffness value and optimize it for better performance by specifically defining the redundant \acrshort{DoF}.

Another important parameter for future research is the collision index. The collision index is used to identify potential collisions between any part of the robot or the end-effector and the workpieces or other objects in the environment. This parameter is particularly significant in scenarios involving \acrshort{WAAM} systems, where loose wires can change their position depending on the robots pose. Preventing collisions is essential to avoid damage to the workpiece, the robot, or other equipment. As of now the optimization algorithm does not consider this option.

One area where the \acrshort{PSO} optimization process can lead to significant advantages is section-wise optimization. The toolpath can be divided into sections, and for each section, the optimal settings of the redundant \acrshort{DoF} can be determined. To make this approach work effectively, it is important to consider the boundary conditions at the transition points between the individual sections.

Furthermore, implementing the proposed method in a \acrshort{CAM} software can provide faster computation and enable the optimization of more complex toolpaths. Another option to speed up the computation is to program the algorithm in such a way that it is designed for multi-threading. Such implementation can result in significantly faster optimization.  

Moving forward, it is necessary to conduct further validation processes, including real-world tests and simulations, to evaluate the performance of the optimized parameters and verify the effectiveness of the proposed methodology. Additionally, exploring the implementation of machine learning techniques for best-case calculations could be a valuable avenue for future research.

%In conclusion, conducting additional tests and thorough validation can greatly enhance the efficiency, productivity, and safety of the manufacturing process.

%Energy Use:
%Redundant degrees of freedom can increase energy consumption in manufacturing systems. Additional joints and actuators require more power, leading to higher energy costs and environmental impact. Efficient energy management strategies, such as optimizing control algorithms, utilizing regenerative braking, and incorporating energy-efficient components, are crucial to minimize energy use while maintaining system performance.